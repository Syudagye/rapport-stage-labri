\chapter{Bilan}
\label{bilan}

Selon moi l'objectif principal de ce stage, qui était de faire un pas dans le monde de la recherche, est un totalement réalisé.
En effet j'envisageait de continuer mes études dans cette voie, mais il me manquait tout de même une part d'expérience, et ce stage était la meilleure opportunité pour pouvoir s'y plonger.

Au cours de ce stage j'ai pu apprendre à correctement lire des articles scientifique pour en tirer des information pertinantes pouvant être utilisées dans le contexte voulu.
Cela m'as aussi donné un aperçu du fonctionnement de la démarche scientifique et des méthodes entrant en jeu dans la mise en place d'expériences, et la visualisation des données.

D'un point de vue technique, je suis satisfait d'avoir pu finir l'instrumentation de CodeGRITS et d'avoir quelque chose qui fonctionne et qui pourrait être utilisé dans le futur par d'autres.
J'ai pu aussi expérimenter avec des outils annexes comme \href{https://asciidoc.org/}{asciidoc} pour l'écriture de mes résumé et documentation, et \LaTeX{} pour ce rapport.

Toutefois, quelques petits manques se font sentirs: tout au début du stage, l'un des objectif potentiels que nous avons évoqué était de mettre en place un protocol expérimental et de pouvoir
le tester avec d'autres stagiaires. Malheureusement, les autres tâches se sont averées plutot chronophages, et c'est donc quelque chose qui n'as pas pû être fait.
Un autre regret est la durée du stage, qui n'as finalement été que d'un mois et demi à cause de limites personnelles et administratives,
ce qui a poser des limites sur ce qui était possible de réaliser et même sur le sujet lui-même.

Je garde tout de même un très bon souvenir de cette période, studieuse et enrichissante, au cours de laquelle j'ai pu rencontrer des personnes inspirantes et decouvrir un monde passionnant.
